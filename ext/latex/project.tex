

\documentclass[a4paper]{report}




\title{Indian Institute of Technology (BHU) Varanasi\\
Department of Computer Science}
\author{Utsav Krishnan}
\date\today

\begin{document}

\maketitle

This is my \emph{first} document prepared in \LaTeX
The numbers 1, 2, 3, etc.\ are called natural numbers. According to Kronecker\cite{a1}, they were made\\[5pt]
by God; all \lq else\rq being \lq\lq the\rq\rq work of Man.
\footnote{footnote}
\footnote{footnote}
\footnote{footnote}
\newpage
\begin{center}
\textbf{\huge Certificate}\\[50pt]
\end{center}

This is to certify that Mr. Utsav Krishnan has undergone a course at this institute and is qualified to be a technician.
\\[20pt]

\begin{minipage}{0.4\textwidth}
\begin{flushleft} \large
\emph{Author:}\\
Utsav Krishnan \\% Your name
\textsc{Part-II, IDD}
\end{flushleft}
\end{minipage}
~
\begin{minipage}{0.55\textwidth}
\begin{flushright} \large
\emph{Supervisor:} \\
Dr. Hari Prabhat\\
\textsc{Gupta} % Supervisor's Name
\end{flushright}
\end{minipage}\\[0cm]


\tableofcontents

\listoffigures

\listoftables

\chapter{Noise removal}
\section{Section}
ssection
\subsection{subsection}
ssubsection
\paragraph{Paragraph}
pparagraph

The average of window size of data points are taken as average and the value is set to the median element. 
\begin{quote}The weight given to every points in the window size over summation is the same.
\end{quote}The quote has ended now.
$$X_\frac{m+n}{2} = \frac{X_m + X_{m+1}+ \cdots + X_n}{n-m+1}
      = \frac{1}{n-m+1}\sum_{i=m}^{n} X_i$$
Data is then normalized to reduce the distance between the maximum and minimum of 
gait cycle points while preserving the characteristics of the graph.\newline

\begin{description}
	\item[AGC] Average gait cycle is the gait cycle which has maximum resemblance to other gait cycles.
	\begin{description}
		\item This syntax is used for definition.
	\end{description}
\end{description}

\begin{verbatim}
   for i from 1 to n
       y[i] := y[i-1] + \alpha * (x[i] - y[i-1])
\end{verbatim}
% bibliography
\begin{thebibliography}{20}
\bibitem{a1}jkvbkjdbvkj
\bibitem{a2}nvdvldknvlskdnvs
\end{thebibliography}


\end{document}
